\begin{enumerate}
    \item Произведем расчет погрешности величин среднего времени падения, ускорения груза, углового ускорения крестовины и момента силы натяжения ниты для массы груза $m_1$ и первой риски:

    $\displaystyle \Delta_t = \sqrt{\frac{\sum_{i = 1}^n (t_i - \overline{t})^2}{3 \cdot 2} + (\frac{2}{3}\Delta_{\text{и}t})^2} = 0.14$

    $\displaystyle \varepsilon_t = \frac{\Delta_t}{\overline{t}} \cdot 100\% = 2.91\%$

    $\displaystyle \Delta_a = \sqrt{\left(\frac{\partial a}{\partial h}\Delta_h\right)^2 + \left(\frac{\partial a}{\partial t} \Delta_t\right)^2} = 
    \sqrt{\Delta_h^2 + \left(-2\frac{\overline{h}}{\overline{t}^3} \Delta_t\right)^2} = \sqrt{0.001^2 + \left(-2\frac{0.7}{4.85}\right)^2} = 0.003$

    $\displaystyle \varepsilon_a = \frac{\Delta_a}{a} \cdot 100\% = 5.1\%$

    $\displaystyle \Delta_\varepsilon = \sqrt{\left(\frac{\partial \varepsilon}{\partial a}\Delta_a\right)^2 + \left(\frac{\partial \varepsilon}{\partial d}\Delta_d\right)^2} = 
    \sqrt{\left(\frac{2}{d}\Delta_a\right)^2 + \left(-\frac{2a}{d^2}\Delta_d\right)^2} = 0.14$

    $\displaystyle \varepsilon_\varepsilon = \frac{\Delta_\varepsilon}{\varepsilon} \cdot 100\% = 5.25\%$

    $\displaystyle \Delta_M = \sqrt{\left(\frac{\partial M}{\partial m}\Delta_m\right)^2 + \left(\frac{\partial M}{\partial d}\Delta_d\right)^2 + \left(\frac{\partial M}{\partial a}\Delta_a\right)^2} = 
    \sqrt{\left(\frac{d(g - a)}{2}\Delta_m\right)^2 + \left(\frac{m(g - a)}{2}\Delta_d\right)^2 + \left(\frac{md}{2}\Delta_a\right)^2} = \\ = 0.001$

    $\displaystyle \varepsilon_M = \frac{\Delta_M}{M} \cdot 100\% = 2.17\%$

    Получаем доверительные интервалы для массы груза $m_1$ и первой риски:

    \begin{tabular}{lll}
        $t_\text{ср} = (4.85 \pm 0.14) \ \text{c}$ & $\varepsilon = 2.91\%$ & $\alpha = 0.95$ \\

        $a = (0.059 \pm 0.003) \ \frac{\text{м}}{\text{с}^2}$ & $\varepsilon = 5.1\%$ & $\alpha = 0.95$ \\

        $\varepsilon = (2.58 \pm 0.14) \ \frac{\text{рад}}{\text{с}^2}$ & $\varepsilon = 5.25\%$ & $\alpha = 0.95$ \\

        $M = (0.060 \pm 0.001) \ \text{Н} \cdot \text{м}$ & $\varepsilon = 2.17\%$ & $\alpha = 0.95$ \\
    \end{tabular}

    \smallvspace

    \item Произведем расчет погрешности величин момента силы трения и момента инерции для утяжелителей на 1 риске, 
    полученных в ходе применения метода наименьших квадратов:

    $\displaystyle \Delta_I = 2 S_I = 2\sqrt{\frac{1}{D} \frac{\sum_{i = 1}^n d_i}{4 - 2}} = 4.67 \cdot 10^{-4}$

    $\displaystyle \varepsilon_I = \frac{\Delta_I}{I} \cdot 100\% = 2.25\%$

    $\displaystyle \Delta_{M_\text{тр}} = 2S_{M_\text{тр}} = 2\sqrt{\left(\frac{1}{n} + \frac{\overline{\varepsilon}^2}{D}\right) \frac{\sum_{i = 1}^n d_i}{4}} = 12.13 \cdot 10^{-4}$

    $\displaystyle \varepsilon_{M_\text{тр}} = \frac{\Delta_{M_\text{тр}}}{{M_\text{тр}}} \cdot 100\% = 20.12\%$

    Получаем доверительные интервалы для первой риски:

    \begin{tabular}{lll}
        $I = (207.62 \cdot 10^{-4} \pm 4.67 \cdot 10^{-4}) \ \text{кг} \cdot \text{м}^2$ & $\varepsilon = 2.25\%$ & $\alpha = 0.95$ \\

        $M_\text{тр} = (60.31 \cdot 10^{-4} \pm 12.13 \cdot 10^{-4}) \ \text{Н} \cdot \text{м}$ & $\varepsilon = 20.12\%$ & $\alpha = 0.95$ \\
    \end{tabular}

    \smallvspace

    \item Произведем расчет погрешности момента инерции стержней крестовин и массы утяжелителей:

    $\displaystyle \Delta_{4m_\text{ут}} = 2 S_{4m_\text{ут}} = 2\sqrt{\frac{1}{D} \frac{\sum_{i = 1}^n d_i}{6 - 2}} = 0.158$

    $\displaystyle \varepsilon_{4m_\text{ут}} = \frac{\Delta_{4m_\text{ут}}}{4m_\text{ут}} \cdot 100\% = 8.40\%$

    $\displaystyle \Delta_{I_0} = 2S_{{I_0}} = 2\sqrt{\left(\frac{1}{n} + \frac{\overline{R^2}^2}{D}\right) \frac{\sum_{i = 1}^n d_i}{4}} = 0.0038$

    $\displaystyle \varepsilon_{I_0} = \frac{\Delta_{I_0}}{I_0} \cdot 100\% = 41.59\%$

    \clearpage

    Получаем доверительные интервалы:

    \begin{tabular}{lll}
        $4m_{\text{ут}} = (1.882 \pm 0.158) \ \text{кг}$ & $\varepsilon = 8.40\%$ & $\alpha = 0.95$ \\

        $m_{\text{ут}} = (0.471 \pm 0.040) \ \text{кг}$ & $\varepsilon = 8.40\%$ & $\alpha = 0.95$ \\

        $I_0 = (0.0092 \pm 0.0038) \ \text{кг} \cdot \text{м}^2$ & $\varepsilon = 41.59\%$ & $\alpha = 0.95$ \\
    \end{tabular}

    \smallvspace



\end{enumerate}