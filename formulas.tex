\begin{center}
    \textbf{Исходные данные}
\end{center}

\begin{flushleft}
    $\text{Масса каретки} \hspace{4.95cm} (47.0 \pm 0.5)\ \text{г}$ \\
    $\text{Масса шайбы} \hspace{5.2cm} (220.0 \pm 0.5)\ \text{г}$ \\
    $\text{Масса грузов на крестовине} \hspace{2.45cm} (408.0 \pm 0.5)\ \text{г}$ \\
    $\text{Расстояние первой риски от оси} \hspace{1.7cm} (57.0 \pm 0.5)\ \text{мм}$ \\
    $\text{Расстояние между рисками} \hspace{2.55cm} (25.0 \pm 0.2)\ \text{мм}$ \\
    $\text{Диаметр ступицы} \hspace{4.35cm} (46.0 \pm 0.5)\ \text{мм}$ \\
    $\text{Диаметр груза на крестовине} \hspace{2.15cm} (40.0 \pm 0.5)\ \text{мм}$ \\
    $\text{Высота груза на крестовине} \hspace{2.4cm} (40.0 \pm 0.5)\ \text{мм}$ \\
\end{flushleft}

\begin{center}
    \textbf{Рабочие формулы}
\end{center}

\text{Ускорение груза}
\[
a = \frac{2h}{t^2}
\]
Где:
\[
\text{$h$ - расстояние, пройденное грузом за время $t$ от начала движения}
\]

\text{Связь между угловым ускорением и линейным ускорением груза} 
\[
\varepsilon = \frac{2a}{d} = \frac{4h}{t^2d}
\]
Где:
\[
\text{$d$ — диаметр ступицы}
\]

\text{Момент силы натяжения нити}
\[
M = \frac{m d}{2} \left(g - a \right) = \frac{m d}{2} \left(g - \frac{2 h}{t^2} \right)
\]

\text{Основной закон динамики вращения}
\[
I \varepsilon = M - M_{\text{тр}}
\]
Где:
\begin{center}
    \text{$I$ — момент инерции крестовины} \\
    \text{$\varepsilon$ — угловое ускорение крестовины} \\
    \text{$M$ и $M_{\text{тр}}$ — моменты сил натяжения нити и трения на крестовине} \\
\end{center}

\begin{center}
    \text{В соответствии с теоремой Штейнера момент инерции крестовины зависит от расстояния} 
    \text{между центрами грузов и осью вращения по формуле}
\end{center} 
\[
I = I_0 + 4 m_{\text{ут}} R^2
\]
Где:
\begin{center}
    \text{$I_0$ - сумма моментов инерции стрежней крестовины, момента инерции ступицы} 
    \text{и собственных центральных моментов инерции утяжелителей}
\end{center}

\text{Момент инерции крестовины с утяжелителями по МНК} 
\[
I = \frac{\sum_{i=1}^{N} \left( \varepsilon_i - \overline{\varepsilon} \right) \left( M_i - \overline{M} \right)}{\sum_{i=1}^{N} \left( \varepsilon_i - \overline{\varepsilon} \right)^2}
\]

\text{Расстояние от оси крестовины до грузов-утяжелителей}: 
\[
R = l_1 + \left( n - 1 \right) l_0 + \frac{b}{2}
\]

Где:
\begin{center}
    \text{$l_1$ – расстояние от оси вращения до первой риски} \\
    \text{$n$ – номер риски, на которой установлены утяжелители} \\
    \text{$l_0$ – расстояние между соседним рисками} \\
    \text{$b$ – размер утяжелителя вдоль спицы} \\
\end{center}

\text{Момент инерции крестовины по т. Штейнера}
\[
I_0 = \overline{I} - 4m_{\text{ут}}\overline{R}^2
\]
Где:
\[
m_\text{ут} = \frac{\sum_{i=1}^{N} \left( R_i - \overline{R} \right) \left( I_i - \overline{I} \right)}{4\sum_{i=1}^{N} \left( R_i - \overline{R} \right)^2}
\]

\[
\Delta m_{\text{ут}} = \frac{2 \cdot \sqrt{\frac{\sum_{i=1}^{N} \left( I_i - \left( I_0 + 4 m_{\text{ут}} R_i^2 \right) \right)^2}{(N - 2) \sum_{i=1}^{N} \left( R_i^2 - \bar{R}^2 \right)}}}{4} \quad \quad \Delta I_0 = 2 \cdot \sqrt{\left( \frac{1}{N} + \frac{\overline{R}^2}{\sum_{i=1}^{N} \left( R_i^2 - \overline{R}^2 \right)^2} \right) \cdot \frac{\sum_{i=1}^{N} \left( I_i - \left( I_0 + 4 m_{\text{ут}} R_i^2 \right) \right)^2}{N-2}}
\]

\text{Абсолютная погрешность с учетом погрешности приборов}: 
\[
\Delta x = \sqrt{\left( \overline{\Delta x} \right)^2 + \left( \frac{2}{3} \Delta_{\text{и}x} \right)^2}
\]
Где:
\begin{center}
    \text{$\Delta_{\text{и}x}$ — погрешность прибора}
    \text{$\overline{\Delta x}$ — случайная погрешность (доверительный интервал)}
\end{center}

\text{Погрешность косвенного значения}: 
\[
\Delta z = \sqrt{\left( \frac{\partial z}{\partial x_1} \Delta x_1 \right)^2 + \left( \frac{\partial z}{\partial x_2} \Delta x_2 \right)^2}
\]
Где:
\[
z = f(x_1, x_2)
\]

\text{Относительная погрешность}: 
\[
\varepsilon_x = \frac{\Delta x}{\overline{x}} \cdot 100\%
\]
